\chapter{Software Requirements Specification (SRS)}

\section{Introduction}
\subsection{Purpose}
This Software Requirements Specification (SRS) document aims to provide a comprehensive description of OpenArmor, including its functionalities, user interfaces, and system requirements. It serves as a guide for developers, testers, and stakeholders throughout the development process.

\subsection{Scope}
OpenArmor is an advanced cybersecurity solution that leverages eBPF logging, AI-driven threat detection, and standardized log formats to enhance an organization's security posture. This document covers the core features, user interactions, and system interfaces of OpenArmor.

\subsection{Definitions, Acronyms, and Abbreviations}
\begin{itemize}
    \item eBPF: extended Berkeley Packet Filter
    \item OCSF: Open Cybersecurity Schema Framework
    \item AI: Artificial Intelligence
    \item ML: Machine Learning
    \item SRS: Software Requirements Specification
\end{itemize}

\section{Overall Description}
\subsection{Product Perspective}
OpenArmor is designed as both a standalone product and an integrated component within larger cybersecurity ecosystems. It enhances existing security infrastructures by providing advanced logging and threat detection capabilities.

\subsection{System Interfaces}
OpenArmor interfaces with:
\begin{itemize}
    \item Operating System Kernel: For eBPF-based logging
    \item Existing SIEM systems: For log ingestion and alert generation
    \item External threat intelligence feeds: For up-to-date threat information
\end{itemize}

\subsection{User Interfaces}
OpenArmor provides:
\begin{itemize}
    \item Web-based dashboard: For real-time monitoring and configuration
    \item Command-line interface: For advanced users and automation
    \item API: For integration with other security tools and custom applications
\end{itemize}

\subsection{Product Functions}
\begin{itemize}
    \item eBPF Logging: Efficient capture of kernel-level system logs with minimal overhead.
    \item OCSF Standardization: Structuring and normalization of logs into standardized formats for interoperability.
    \item Kernel Space Logging: Extraction of logs directly from the kernel space, providing lower-level visibility.
    \item AI Log Processing: Parsing, analyzing, and transforming logs into standardized formats using artificial intelligence algorithms.
    \item Automated Threat Detection: Utilization of machine learning to baseline normal behavior and identify anomalies indicative of cyber threats.
    \item Alert Generation: Creation and prioritization of security alerts based on detected anomalies.
    \item Reporting: Generation of detailed security reports and visualizations.
\end{itemize}

\section{User Classes and Characteristics}
\subsection{Cybersecurity Analysts}
\begin{itemize}
    \item Primary users of the system
    \item High level of technical expertise in cybersecurity
    \item Require detailed threat information and advanced analysis tools
\end{itemize}

\subsection{System Administrators}
\begin{itemize}
    \item Responsible for deployment and maintenance of OpenArmor
    \item Strong technical background in system administration
    \item Need configuration and performance monitoring tools
\end{itemize}

\subsection{IT Managers}
\begin{itemize}
    \item Oversee cybersecurity operations
    \item Require high-level dashboards and summary reports
    \item Less technical, more focused on strategic decision-making
\end{itemize}

\section{Operating Environment}
\subsection{Hardware Requirements}
\begin{itemize}
    \item Minimum: 8-core CPU, 16GB RAM, 500GB SSD
    \item Recommended: 16-core CPU, 32GB RAM, 1TB SSD
\end{itemize}

\subsection{Software Requirements}
\begin{itemize}
    \item Operating System: Linux (kernel version 4.15 or later)
    \item Database: PostgreSQL 12 or later
    \item Web Server: Nginx or Apache
\end{itemize}

\subsection{Network Requirements}
\begin{itemize}
    \item Gigabit Ethernet connection
    \item Outbound internet access for threat intelligence updates
\end{itemize}

\section{Design and Implementation Constraints}
\begin{itemize}
    \item Must comply with relevant data protection regulations (e.g., GDPR, CCPA)
    \item Should be scalable to handle large enterprise environments
    \item Must support high availability and disaster recovery configurations
\end{itemize}

\section{Assumptions and Dependencies}
\subsection{Assumptions}
\begin{itemize}
    \item Users have basic familiarity with cybersecurity concepts
    \item The operating environment supports eBPF technology
    \item Consistent internet connectivity for threat intelligence updates
\end{itemize}

\subsection{Dependencies}
\begin{itemize}
    \item Relies on up-to-date threat intelligence feeds
    \item Depends on machine learning libraries for AI-driven analysis
    \item Requires regular updates to maintain effectiveness against evolving threats
\end{itemize}

\section{External Interface Requirements}

\subsection{User Interfaces}
OpenArmor's user interface will be a web-based dashboard, similar to Wazuh's, providing:
\begin{itemize}
    \item Real-time event viewer with filtering and search capabilities
    \item Interactive visualizations for system and security metrics
    \item Configuration management interface for agents and rules
    \item Alert management and investigation tools
    \item Customizable dashboards for different user roles
\end{itemize}

\subsection{Hardware Interfaces}
OpenArmor will support various hardware sensors and security appliances, including:
\begin{itemize}
    \item Network Interface Cards (NICs) for packet capture
    \item Hardware Security Modules (HSMs) for secure key storage
    \item IPMI-enabled devices for out-of-band management
\end{itemize}

\subsection{Software Interfaces}
OpenArmor will integrate with and extend the capabilities of:
\begin{itemize}
    \item Wazuh: For host-based intrusion detection and log analysis
    \item OSquery: For querying endpoint state information
    \item Sysmon: For detailed Windows event logging
    \item SIEM systems: Via standardized log formats (OCSF)
    \item Threat Intelligence Platforms: For up-to-date IoCs and threat data
\end{itemize}

APIs will be provided for:
\begin{itemize}
    \item RESTful data access and management
    \item Webhook integrations for alerts and events
    \item Custom plugin development
\end{itemize}

\subsection{Communications Interfaces}
OpenArmor will support:
\begin{itemize}
    \item Encrypted agent-server communication (similar to Wazuh)
    \item HTTPS for web interface access
    \item SSH for remote management
    \item Syslog for log ingestion
    \item MQTT for IoT device communication
\end{itemize}

\section{Functional Requirements}

\subsection{eBPF Logging}
\begin{itemize}
    \item Capture system calls, network events, and file operations
    \item Provide real-time streaming of eBPF events
    \item Allow custom eBPF programs for specialized monitoring
\end{itemize}

\subsection{OCSF Standardization}
\begin{itemize}
    \item Convert logs from various sources (Wazuh, OSquery, Sysmon) to OCSF format
    \item Provide mapping tools for custom log sources
    \item Ensure compatibility with OCSF-compliant SIEM systems
\end{itemize}

\subsection{Kernel Space Logging}
\begin{itemize}
    \item Integrate Sysmon-style detailed event logging for Windows systems
    \item Develop Linux kernel module for enhanced logging capabilities
    \item Provide kernel-level visibility without performance impact
\end{itemize}

\subsection{AI Log Processing}
\begin{itemize}
    \item Implement machine learning models for log parsing and normalization
    \item Develop AI-driven correlation engine for complex event analysis
    \item Provide automated log summarization and insights
\end{itemize}

\subsection{Automated Threat Detection}
\begin{itemize}
    \item Integrate and enhance Wazuh's rule-based detection capabilities
    \item Implement anomaly detection using machine learning models
    \item Provide behavior-based detection for advanced persistent threats
\end{itemize}

\subsection{OSquery Integration}
\begin{itemize}
    \item Incorporate OSquery for on-demand and scheduled system state queries
    \item Extend OSquery capabilities with custom tables for eBPF data
    \item Provide a unified interface for querying data from all monitored systems
\end{itemize}

\section{Non-Functional Requirements}

\subsection{Performance Requirements}
\begin{itemize}
    \item Process up to 100,000 events per second on recommended hardware
    \item Web interface response time under 2 seconds for most operations
    \item Agent resource usage below 5%- CPU and 256MB RAM on monitored systems
\end{itemize}

\subsection{Safety Requirements}
\begin{itemize}
    \item Implement safeguards to prevent eBPF programs from crashing the kernel
    \item Ensure that log collection doesn't interfere with critical system operations
    \item Provide failsafe mechanisms for agents to prevent system instability
\end{itemize}

\subsection{Security Requirements}
\begin{itemize}
    \item End-to-end encryption for all communications
    \item Role-based access control (RBAC) for user management
    \item Multi-factor authentication for administrative access
    \item Secure key management for agent-server communications
    \item Regular security audits and penetration testing
\end{itemize}

\subsection{Software Quality Attributes}
\begin{itemize}
    \item Availability: 99.99%- uptime for core services
    \item Scalability: Support for up to 100,000 endpoints in a single deployment
    \item Maintainability: Modular architecture for easy updates and extensions
    \item Interoperability: Standard APIs and data formats for integration with existing security tools
\end{itemize}

\section{Conclusion}
This SRS provides a comprehensive overview of the OpenArmor system, incorporating key features from Wazuh, OSquery, and Sysmon while extending their capabilities with eBPF and AI-driven analysis. It serves as a foundation for the development process and should be updated as the project evolves. The integration of these established tools with OpenArmor's innovative features aims to create a powerful, unified cybersecurity solution capable of addressing the complex threat landscape faced by modern organizations.