\chapter{Literature Survey}

\section{Introduction}
This chapter presents a comprehensive review of recent research relevant to the development of OpenArmor. The survey covers various aspects of cybersecurity, including eBPF technology, AI-driven security solutions, event extraction, monitoring systems, and semantic web approaches for cybersecurity information management. Each paper is summarized and its relevance to OpenArmor is discussed.

\section{Advanced Network Functions and Monitoring}
\subsection{eBPF-Based Network Functions}
\textbf{Title}: A Framework for eBPF-Based Network Functions in an Era of Microservices 
\textbf{Authors}: Miano, S., Risso, F., Bernal, M. V., Bertrone, M., and Lu, Y. (2021)
\textbf{Publication}: IEEE Transactions on Network and Service Management, 18(1), 133-151

\textbf{Summary}: [Your existing summary]

\textbf{Relevance to OpenArmor}: This paper's framework for eBPF-based network functions aligns closely with OpenArmor's use of eBPF for advanced system activity logging. The high performance and flexibility demonstrated in this research validate the choice of eBPF technology for OpenArmor's core logging functionality.

\subsection{Distributed Cloud Monitoring}
\textbf{Title}: Distributed cloud monitoring using Docker as next generation container virtualization technology
\textbf{Authors}: Dhakate, S., and Godbole, A. (2015)
\textbf{Publication}: 2015 Annual IEEE India Conference (INDICON) (pp. 1-5)

\textbf{Summary}: [Your existing summary]

\textbf{Relevance to OpenArmor}: The distributed monitoring approach using containerization technology could inform OpenArmor's architecture, especially for deploying and managing monitoring components across diverse environments.

\section{AI and Machine Learning in Cybersecurity}
\subsection{AI-Driven Cybersecurity Overview}
\textbf{Title}: AI‑Driven Cybersecurity: An Overview, Security Intelligence Modeling and Research Directions
\textbf{Authors}: Sarker, I. H., Furhad, M. H., and Nowrozy, R. (2021)
\textbf{Publication}: SN Computer Science, 2(3), 173

\textbf{Summary}: [Your existing summary]

\textbf{Relevance to OpenArmor}: This paper provides a comprehensive overview of AI applications in cybersecurity, which is directly relevant to OpenArmor's AI-driven threat detection and analysis capabilities. The research directions outlined could guide future enhancements of OpenArmor.

\section{Cybersecurity Event Extraction and Analysis}
\subsection{Document-level Cybersecurity Event Extraction}
\textbf{Title}: A Framework for Document-level Cybersecurity Event Extraction from Open Source Data
\textbf{Authors}: Luo, N., Du, X., He, Y., Jiang, J., Wang, X., Jiang, Z., and Zhang, K. (2021)
\textbf{Publication}: 2021 IEEE 24th International Conference on Computer Supported Cooperative Work in Design (CSCWD) (pp. 422-427)

\textbf{Summary}: [Your existing summary]

\textbf{Relevance to OpenArmor}: The event extraction framework presented in this paper could enhance OpenArmor's ability to process and analyze unstructured data sources, improving threat intelligence capabilities.

\subsection{Rich Semantic Information Extraction}
\textbf{Title}: Extracting rich semantic information about cybersecurity events
\textbf{Authors}: Satyapanich, T., Finin, T., and Ferraro, F. (2019)
\textbf{Publication}: 2019 IEEE International Conference on Big Data (Big Data) (pp. 5034-5042)

\textbf{Summary}: [Your existing summary]

\textbf{Relevance to OpenArmor}: This research could inform the development of OpenArmor's data processing pipeline, enabling more comprehensive and semantically rich threat intelligence extraction from diverse data sources.


\section{Research Paper 1}
\textbf{Title }: A Framework for eBPF-Based Network Functions in an Era of Microservices 
\\
\textbf{Author }: Miano, S., Risso, F., Bernal, M. V., Bertrone, M., and Lu, Y. (2021). A framework for eBPF-based network functions in an era of microservices. IEEE Transactions on Network and Service Management, 18(1), 133-151.
\\
\textbf{Summary }: The paper proposes a framework that leverages eBPF (extended Berkeley Packet Filter) technology to develop and deploy network functions as eBPF programs in microservices environments. The framework consists of components for eBPF program development, deployment, and communication, enabling efficient and scalable implementation of network functions like load balancers and firewalls. Evaluation results demonstrate the framework's ability to achieve high throughput and low latency, comparable or better than traditional kernel-bypass solutions, while offering improved flexibility and agility in provisioning network functions.

\section{Research Paper 2}
\textbf{Title} : A Framework for Document-level Cybersecurity Event Extraction from Open Source Data
\\
\textbf{Author} : Luo, N., Du, X., He, Y., Jiang, J., Wang, X., Jiang, Z., and Zhang, K. (2021, May). A framework for document-level cybersecurity event extraction from open source data. In 2021 IEEE 24th International Conference on Computer Supported Cooperative Work in Design (CSCWD) (pp. 422-427). IEEE.
\\
\newpage
\textbf{Summary} : The paper presents a framework for extracting cybersecurity events from open-source data at the document level. It proposes a deep learning model that performs joint entity recognition and event extraction, capturing both intra- and inter-sentence dependencies. The framework leverages external knowledge bases to enrich the extracted events with contextual information. Experimental results on real-world datasets demonstrate the framework's effectiveness in accurately identifying and characterizing cybersecurity incidents from unstructured text data.

\section{Research Paper 3}
\textbf{Title }: Distributed cloud monitoring using Docker as next generation container virtualization technology
\\
\textbf{Author }: Dhakate, S., and Godbole, A. (2015, December). Distributed cloud monitoring using Docker as next generation container virtualization technology. In 2015 Annual IEEE India Conference (INDICON) (pp. 1-5). IEEE.
\\
\textbf{Summary }: This paper proposes a distributed cloud monitoring system that leverages Docker, a next-generation container virtualization technology. The system employs Docker containers to encapsulate monitoring agents, enabling efficient deployment and management of monitoring components across distributed cloud environments. The authors demonstrate the system's ability to monitor cloud resources effectively while minimizing overhead and providing scalability benefits compared to traditional virtualization approaches.

\section{Research Paper 4}
\textbf{Title} : AI‑Driven Cybersecurity: An Overview, Security Intelligence Modeling and Research Directions
\\
\textbf{Author} : Sarker, I. H., Furhad, M. H., and Nowrozy, R. (2021). Ai-driven cybersecurity: an overview, security intelligence modeling and research directions. SN Computer Science, 2(3), 173.
\\
\textbf{Summary} : This paper provides an overview of leveraging artificial intelligence (AI) for cybersecurity. It explores various AI and machine learning techniques like deep learning, reinforcement learning, and ensemble methods that can be applied to domains such as network security, malware detection, and intrusion prevention. The authors highlight the benefits of AI-driven security solutions, including adaptability, scalability, and proactive threat detection capabilities. The paper also outlines research challenges and future directions for developing robust AI-based cybersecurity systems, such as handling adversarial attacks, dealing with data scarcity, and ensuring model transparency and interpretability.

\section{Research Paper 5}
\textbf{Title} : Unpacking strategic behavior in cyberspace: a schema-driven approach
\\
\textbf{Author} : Gomez, M. A., and Whyte, C. (2022). Unpacking strategic behavior in cyberspace: a schema-driven approach. Journal of Cybersecurity, 8(1), tyac005.
\\
\textbf{Summary} : This paper proposes a schema-driven approach to analyze and understand strategic behavior in cyberspace. It introduces a framework that combines cognitive schemas and game theory to model the decision-making processes and interactions between adversaries in cyber conflicts. The authors argue that this approach can provide insights into the motivations, goals, and potential actions of cyber threat actors, enabling more effective cybersecurity strategies and deterrence mechanisms. The framework is illustrated through case studies, demonstrating its applicability in unpacking the complexities of strategic cyber behavior.

\section{Research Paper 6}
\textbf{Title} : Developing a UI and Automation Framework for a Cybersecurity Research and Experimentation Environment
\\
\textbf{Author} : Butler, C., Thompson, G., Hsieh, G., Hoppa, M. A., and Nauer, K. S. (2018). Developing a UI and Automation Framework for a Cybersecurity Research and Experimentation Environment. In Proceedings of the International Conference on Security and Management (SAM) (pp. 208-213). The Steering Committee of The World Congress in Computer Science, Computer Engineering and Applied Computing (WorldComp).
\\
\textbf{Summary} : The paper describes the development of a user interface (UI) and automation framework for a cybersecurity research and experimentation environment. The framework aims to simplify the process of configuring and deploying cybersecurity experiments, enabling researchers to focus on their core objectives. It provides a web-based UI for defining experiment parameters and orchestrating the deployment of virtual machines and network configurations. The automation capabilities streamline the setup, execution, and data collection phases of cybersecurity experiments, enhancing productivity and reproducibility.

\newpage
\section{Research Paper 7}
\textbf{Title} : Extracting rich semantic information about cybersecurity events
\\
\textbf{Author} : Satyapanich, T., Finin, T., and Ferraro, F. (2019, December). Extracting rich semantic information about cybersecurity events. In 2019 IEEE International Conference on Big Data (Big Data) (pp. 5034-5042). IEEE.
\\
\textbf{Summary} : This paper presents an approach for extracting rich semantic information about cybersecurity events from unstructured text data sources. The authors propose a hybrid system that combines machine learning techniques with knowledge-based methods to identify and characterize cybersecurity incidents. Their system utilizes named entity recognition, relation extraction, and event detection models to extract relevant entities, relationships, and event details from text. The extracted information is then represented using semantic web technologies, enabling complex querying and reasoning over cybersecurity event data. Evaluation on real-world datasets demonstrates the system's effectiveness in accurately capturing comprehensive details about cybersecurity incidents from textual reports.

\section{Research Paper 8}
\textbf{Title} : An autonomous cybersecurity framework for next-generation digital service chains
\\
\textbf{Author} : Repetto, M., Striccoli, D., Piro, G., Carrega, A., Boggia, G., and Bolla, R. (2021). An autonomous cybersecurity framework for next-generation digital service chains. Journal of Network and Systems Management, 29(4), 37.
\\
\textbf{Summary} : This paper proposes an autonomous cybersecurity framework for securing next-generation digital service chains in 5G and beyond networks. The framework employs machine learning techniques and software-defined networking principles to dynamically deploy and orchestrate virtual security functions based on detected threats and service requirements. It enables proactive and adaptive security management, automating the provisioning of security services while optimizing resource utilization. The authors evaluate the framework's performance, demonstrating its ability to provide effective and efficient cybersecurity protection for complex service chains.

\newpage
\section{Research Paper 9}
\textbf{Title} : Integrating Cybersecurity Into a Big Data Ecosystem
\\
\textbf{Author} : Tall, A. M., Zou, C. C., and Wang, J. (2021, November). Integrating Cybersecurity Into a Big Data Ecosystem. In MILCOM 2021-2021 IEEE Military Communications Conference (MILCOM) (pp. 69-76). IEEE.
\\
\textbf{Summary} : This paper presents an approach to integrate cybersecurity capabilities into a big data ecosystem. The authors propose a framework that leverages big data technologies and machine learning techniques to process and analyze large volumes of security data from various sources. The framework aims to provide real-time threat detection, risk assessment, and incident response capabilities within a unified big data platform, enabling efficient and scalable cybersecurity operations.

\section{Research Paper 10}
\textbf{Title} : Web of cybersecurity: Linking, locating, and discovering structured cybersecurity information
\\
\textbf{Author} : Takahashi, T., Panta, B., Kadobayashi, Y., and Nakao, K. (2018). Web of cybersecurity: Linking, locating, and discovering structured cybersecurity information. International Journal of Communication Systems, 31(3), e3470.
\\
\textbf{Summary} : The paper introduces the concept of a "Web of Cybersecurity," a decentralized network for sharing and discovering structured cybersecurity information. The authors propose a linked data approach, where cybersecurity data is represented using semantic web technologies and interconnected through links. This enables efficient discovery, integration, and analysis of cybersecurity information from diverse sources. The paper outlines techniques for linking, locating, and querying cybersecurity data within this web-based ecosystem.


\section{Conclusion}
This literature survey has highlighted several key areas of research relevant to OpenArmor's development:

1. Advanced network function implementation using eBPF technology
2. AI and machine learning applications in cybersecurity
3. Cybersecurity event extraction and analysis from unstructured data
4. Distributed monitoring and big data integration for cybersecurity
5. Semantic web approaches for cybersecurity information management

These studies provide valuable insights and methodologies that can inform the design and implementation of OpenArmor's core features, including its advanced logging system, AI-driven threat detection, and proactive alert mechanisms. Future development of OpenArmor should consider incorporating the novel approaches and addressing the challenges identified in this survey.
