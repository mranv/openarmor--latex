\chapter{Methodology}

The development of OpenArmor, an advanced cybersecurity solution leveraging AI and advanced logging techniques, followed a systematic approach that integrates cutting-edge technologies with established security tools. Our methodology can be divided into four main phases: data acquisition, data integration and normalization, data processing and analysis, and threat detection and response.

\section{Data Acquisition Phase}

This phase focused on gathering comprehensive system and network data from multiple sources:

\subsection{eBPF-based Kernel Monitoring}
We implemented eBPF (Extended Berkeley Packet Filter) programs for efficient kernel-level logging and monitoring of system activities. eBPF provides:
\begin{itemize}
    \item Low-overhead, real-time monitoring of system calls, network events, and file operations
    \item Safe execution of sandboxed programs in the Linux kernel
    \item Customizable data collection points for comprehensive visibility
\end{itemize}

\subsection{Integration with Existing Security Tools}
We leveraged the capabilities of established security tools:
\begin{itemize}
    \item Wazuh: For host-based intrusion detection and file integrity monitoring
    \item OSquery: To query endpoint state information using SQL-like syntax
    \item Sysmon: For detailed Windows event logging and process monitoring
\end{itemize}

\subsection{Network Traffic Analysis}
We implemented network traffic capture and analysis using:
\begin{itemize}
    \item libpcap for efficient packet capture
    \item Deep packet inspection techniques for protocol analysis
    \item NetFlow/IPFIX collection for network flow monitoring
\end{itemize}

\section{Data Integration and Normalization Phase}

In this phase, we focused on centralizing and standardizing the diverse data sources:

\subsection{OCSF Standardization}
We adopted the OCSF (Open Cybersecurity Schema Framework) to structure and normalize log data:
\begin{itemize}
    \item Developed custom parsers for eBPF, Wazuh, OSquery, and Sysmon data
    \item Implemented OCSF schema mapping and validation
    \item Created an extensible framework for adding new data source adapters
\end{itemize}

\subsection{Data Enrichment}
We enriched the normalized data with contextual information:
\begin{itemize}
    \item Geo-location data for IP addresses
    \item Threat intelligence feed integration for known IoCs
    \item Asset management integration for system context
\end{itemize}

\section{Data Processing and Analysis Phase}

This phase involved preparing the data for analysis and implementing advanced analytics:

\subsection{Data Preprocessing}
We applied various preprocessing techniques to ensure data quality:
\begin{itemize}
    \item Feature extraction and selection
    \item Handling of missing data and outliers
    \item Data normalization and scaling
\end{itemize}

\subsection{Machine Learning Pipeline}
We developed a comprehensive machine learning pipeline:
\begin{itemize}
    \item Unsupervised learning for anomaly detection:
    \begin{itemize}
        \item Isolation Forest for outlier detection
        \item DBSCAN for density-based clustering of security events
    \end{itemize}
    \item Supervised learning for threat classification:
    \begin{itemize}
        \item Random Forest for multi-class threat categorization
        \item Gradient Boosting for binary classification of malicious/benign events
    \end{itemize}
    \item Deep learning models for complex pattern recognition:
    \begin{itemize}
        \item LSTM networks for sequence-based anomaly detection in log data
        \item Autoencoders for dimensionality reduction and feature learning
    \end{itemize}
\end{itemize}

\subsection{Real-time Analytics}
We implemented streaming analytics capabilities:
\begin{itemize}
    \item Apache Kafka for high-throughput event streaming
    \item Apache Flink for real-time data processing and analytics
    \item Custom sliding window algorithms for time-series analysis
\end{itemize}

\section{Threat Detection and Response Phase}

This phase focused on identifying threats and facilitating rapid response:

\subsection{Automated Threat Detection}
We developed a multi-layered threat detection system:
\begin{itemize}
    \item Rule-based detection using Wazuh's capabilities
    \item Anomaly-based detection using our machine learning models
    \item Behavior-based detection for identifying complex attack patterns
\end{itemize}

\subsection{Alert Prioritization and Triage}
We implemented an intelligent alert management system:
\begin{itemize}
    \item Risk scoring algorithm considering threat severity and asset criticality
    \item Alert correlation to identify related security events
    \item Automated alert enrichment with contextual information
\end{itemize}

\subsection{Automated Response}
We developed capabilities for automated threat response:
\begin{itemize}
    \item Integration with firewall and EDR solutions for automated blocking
    \item Customizable playbooks for orchestrating response actions
    \item AI-assisted decision support for complex incident response
\end{itemize}

\section{Continuous Improvement}

Throughout the development process, we implemented mechanisms for continuous improvement:
\begin{itemize}
    \item Regular model retraining to adapt to evolving threats
    \item A/B testing of detection algorithms to optimize performance
    \item Feedback loops from security analysts to improve alert quality
    \item Integration of emerging threat intelligence to enhance detection capabilities
\end{itemize}

\section{Conclusion}

The methodology employed in developing OpenArmor combines advanced technologies like eBPF and AI with the strengths of established security tools such as Wazuh, OSquery, and Sysmon. By following this comprehensive approach, we've created a robust, adaptable, and intelligent cybersecurity solution capable of providing enterprise-grade protection through continuous monitoring, automated analysis, and timely response to emerging threats.